\documentclass{article}
\usepackage{amsmath,amssymb,amsthm,latexsym,paralist}

\theoremstyle{definition}
\newtheorem{problem}{Problem}
\newtheorem*{solution}{Solution}
\newtheorem*{resources}{Resources}

\newcommand{\name}[2]{\noindent\textbf{Name: #1}\hfill \textbf{UIN: #2}}
\newcommand{\honor}{\noindent On my honor, as an Aggie, I have neither
  given nor received any unauthorized aid on any portion of the
  academic work included in this assignment. Furthermore, I have
  disclosed all resources (people, books, web sites, etc.) that have
  been used to prepare this homework. \\[2ex]
 \textbf{Electronic signature:} \underline{ \textbf{(type your full name here)} } }
 
\newcommand{\checklist}{\noindent\textbf{Checklist:}
\begin{compactitem}[$\Box$] 
\item Did you type in your name and UIN? 
\item Did you disclose all resources that you have used? \\
(This includes all people, books, websites, etc.\ that you have consulted.)
\item Did you sign that you followed the Aggie Honor Code? 
\item Did you solve all problems? 
\item Did you submit the .tex and .pdf files of your homework to the correct link on Canvas? 
\end{compactitem}
}

\newcommand{\problemset}[1]{\begin{center}\textbf{Problem Set #1}\end{center}}
\newcommand{\duedate}[1]{\begin{quote}\textbf{Due dates:} Electronic
    submission of \textsl{yourLastName-yourFirstName-hw6.tex} and 
    \textsl{yourLastName-yourFirstName-hw6.pdf} files of this homework is due on
    \textbf{#1} on \texttt{https://canvas.tamu.edu}. You will see two separate links
    to turn in the .tex file and the .pdf file separately. Please do not archive or compress the files.  
    \textbf{If any of the two files are missing or unreadable, you will receive zero points for this
    homework.}\end{quote} }

\newcommand{\N}{\mathbf{N}}
\newcommand{\R}{\mathbf{R}}
\newcommand{\Z}{\mathbf{Z}}


\begin{document}
\vspace*{-20mm}
\begin{center}
{\large
CSCE 222 Discrete Structures for Computing -- Fall 2021\\[.5ex]
Hyunyoung Lee\\}
\end{center}
\problemset{6}
\duedate{Monday, 10/19/2021 11:59 p.m.}
\name{ (type your name here) }{ (type your UIN here) }
\begin{resources} (All people, books, articles, web pages, etc.\ that
  have been consulted when producing your answers to this homework)
\end{resources}
\honor

\bigskip

\noindent
Total 100 points.

\bigskip

\noindent
The problems are from the lecture notes posted on Perusall.

\medskip

\noindent
All five questions are proof-by-induction questions.  For each proof, strictly adhere 
to the format and the style we discussed in class.  Especially in the inductive step, 
carefully explain each step-by-step in your own words!  You will not receive full credit 
if you don't explain the steps.

Grading rubrics for each induction proof are as follows: 20\% for the induction base
(clearly stating the base case and showing that the base case holds), 70\% for the
inductive step (correctly stating the induction hypothesis, clearly stating where
and how the induction hypothesis is used in the rest of the inductive step, clearly
and completely showing the derivation steps in the inductive step), and 10\% for the 
concluding remark of your proof. 

\medskip

\begin{problem} (20 points) Section 4.1, Exercise 4.3
\end{problem}
\begin{solution} 
\end{solution}

\begin{problem} (20 points) Section 4.1, Exercise 4.4
\end{problem}
\begin{solution} 
\end{solution}

\begin{problem} (20 points) Section 4.1, Exercise 4.6
\end{problem}
\begin{solution} 
\end{solution}

\begin{problem} (20 points) Section 4.3, Exercise 4.15
\end{problem}
\begin{solution} 
\end{solution}

\begin{problem} (20 points) Section 4.3, Exercise 4.17
\end{problem}
\begin{solution} 
\end{solution}

\goodbreak
\checklist
\end{document}
